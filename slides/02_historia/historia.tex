\documentclass[aspectratio=43,handout]{beamer}
% \documentclass[aspectratio=169]{beamer}

% Title --------------------------------------------
\title{\huge El mundo de hoy}
\input{../beamer_preamble.tex}

\begin{document}

%------------------------------------------------
\begin{frame}
\titlepage
\end{frame}

%------------------------------------------------
\begin{frame}{El problema central}
\begin{itemize}
    \item ¿Por qué el mundo ha alternado entre largos períodos de paz y conflictos catastróficos?
    \item ¿Por qué la globalización se expandió en algunas épocas y colapsó en otras?
    \item ¿Cómo influyen el poder, los intereses y las instituciones en los resultados internacionales?
\end{itemize}
\end{frame}

%------------------------------------------------
\begin{frame}{Un enfoque analítico}
\begin{block}{Preguntas clave}
\begin{itemize}
    \item ¿Qué \textbf{intereses} están en juego?
    \item ¿Cómo \textbf{interactúan} los Estados (conflicto vs. cooperación)?
    \item ¿Qué \textbf{instituciones} estructuran estas interacciones?
\end{itemize}
\end{block}
\end{frame}

%------------------------------------------------
\begin{frame}{Un gran contraste histórico}
\begin{itemize}
    \item 1815–1914: Paz relativa, crecimiento y globalización
    \item 1914–1945: Guerras mundiales, depresión y colapso
    \item Posguerra de 1945: Nuevo orden mundial bajo la rivalidad EE. UU.–URSS
\end{itemize}
\end{frame}

%------------------------------------------------
\section{La era mercantilista}

\begin{frame}{Antes de 1500: Un mundo fragmentado}
\begin{itemize}
    \item Interacción limitada a larga distancia
    \item El comercio existía, pero era costoso y poco frecuente
    \item No existía un sistema político ni económico global
\end{itemize}
\end{frame}

%------------------------------------------------
\begin{frame}{La expansión europea después de 1500}
\begin{itemize}
    \item Exploración, conquista y colonización
    \item Los Estados europeos dominan la política mundial
    \item Emergencia de una economía mundial
\end{itemize}
\end{frame}

%------------------------------------------------
\begin{frame}{El mercantilismo}
\begin{definition}
Doctrina económica que vincula el poder militar con la riqueza económica
\end{definition}
\begin{itemize}
    \item Las colonias sirven a los intereses de la metrópoli
    \item Monopolios comerciales y restricciones
    \item La riqueza financia la guerra; la guerra asegura la riqueza
\end{itemize}
\end{frame}

%------------------------------------------------
\begin{frame}{La guerra y el nacimiento del sistema de Estados}
\begin{itemize}
    \item Rivalidad constante entre las potencias europeas
    \item Guerra de los Treinta Años (1618–1648)
    \item Paz de Westfalia: soberanía y no intervención
\end{itemize}
\end{frame}

%------------------------------------------------
\section{La Pax Britannica}

\begin{frame}{El fin del mercantilismo}
\begin{itemize}
    \item La Revolución Industrial transforma los intereses
    \item Gran Bretaña lidera el giro hacia el libre comercio
    \item Los mercados sustituyen a los monopolios
\end{itemize}
\end{frame}

%------------------------------------------------
\begin{frame}{La Pax Britannica (1815–1914)}
\begin{itemize}
    \item La hegemonía británica estabiliza Europa
    \item Diplomacia del equilibrio de poder
    \item Pocas guerras entre grandes potencias
\end{itemize}
\end{frame}

%------------------------------------------------
\begin{frame}{La primera era de la globalización}
\begin{itemize}
    \item Rápida expansión del libre comercio
    \item Migraciones masivas y flujos de capital
    \item Revoluciones tecnológicas en transporte y comunicaciones
\end{itemize}
\end{frame}

%------------------------------------------------
\begin{frame}{El patrón oro}
\begin{itemize}
    \item Tipos de cambio fijos vinculados al oro
    \item Estabilidad monetaria que facilita el comercio
    \item Alta integración, pero frágil
\end{itemize}
\end{frame}

%------------------------------------------------
\begin{frame}{Colonialismo sin mercantilismo}
\begin{itemize}
    \item Renovada expansión imperial después de 1870
    \item África y Asia colonizadas
    \item Creciente resistencia nacionalista
\end{itemize}
\end{frame}

%------------------------------------------------
\section{La crisis de treinta años}

\begin{frame}{La ruptura tras 1914}
\begin{itemize}
    \item Tensiones por cambios en el equilibrio de poder (Alemania)
    \item Endurecimiento de los sistemas de alianzas
    \item Estalla la Primera Guerra Mundial
\end{itemize}
\end{frame}

%------------------------------------------------
\begin{frame}{La Primera Guerra Mundial: consecuencias}
\begin{itemize}
    \item Costes humanos masivos
    \item Colapso de imperios
    \item Estados nuevos e inestables
\end{itemize}
\end{frame}

%------------------------------------------------
\begin{frame}{La inestabilidad de entreguerras}
\begin{itemize}
    \item Reparaciones y resentimiento
    \item Hiperinflación y Gran Depresión
    \item Ascenso del fascismo y del comunismo
\end{itemize}
\end{frame}

%------------------------------------------------
\begin{frame}{La Segunda Guerra Mundial}
\begin{itemize}
    \item Guerra total a escala global
    \item Eje contra Aliados
    \item Destrucción sin precedentes
\end{itemize}
\end{frame}

%------------------------------------------------
\section{El mundo después de la Segunda Guerra Mundial}

\begin{frame}{Un mundo transformado}
\begin{itemize}
    \item Europa y Japón devastados
    \item Solo dos superpotencias reales
    \item Estados Unidos y la Unión Soviética
\end{itemize}
\end{frame}

%------------------------------------------------
\begin{frame}{La Guerra Fría}
\begin{itemize}
    \item Conflicto ideológico: capitalismo vs. comunismo
    \item Enfrentamiento militar sin guerra directa
    \item Competencia a escala global
\end{itemize}
\end{frame}

%------------------------------------------------
\begin{frame}{El bloque occidental: liderazgo estadounidense}
\begin{itemize}
    \item OTAN: seguridad colectiva
    \item Plan Marshall: reconstrucción
    \item Orden económico de Bretton Woods
\end{itemize}
\end{frame}

%------------------------------------------------
\begin{frame}{El sistema de Bretton Woods}
\begin{itemize}
    \item FMI, Banco Mundial, GATT
    \item Globalización gestionada
    \item Compromiso con el Estado de bienestar
\end{itemize}
\end{frame}

%------------------------------------------------
\begin{frame}{El bloque oriental: el modelo soviético}
\begin{itemize}
    \item Planificación central
    \item Pacto de Varsovia
    \item Consejo de Ayuda Mutua Económica
\end{itemize}
\end{frame}

%------------------------------------------------
\begin{frame}{La disuasión nuclear}
\begin{itemize}
    \item Destrucción Mutua Asegurada (MAD)
    \item Ausencia de guerra directa entre superpotencias
    \item Riesgo constante de catástrofe
\end{itemize}
\end{frame}

%------------------------------------------------
\begin{frame}{Crisis de la Guerra Fría}
\begin{itemize}
    \item Bloqueo de Berlín
    \item Guerra de Corea
    \item Crisis de los misiles en Cuba
\end{itemize}
\end{frame}

%------------------------------------------------
\begin{frame}{Descolonización y Guerra Fría}
\begin{itemize}
    \item Colapso de los imperios europeos
    \item Nuevos Estados buscan autonomía
    \item Competencia entre superpotencias en el Sur Global
\end{itemize}
\end{frame}

%------------------------------------------------
\begin{frame}{Conclusión}
\begin{itemize}
    \item La política internacional está marcada por cambios en el poder
    \item Las instituciones importan, pero son frágiles
    \item El orden posterior a 1945 evitó una guerra mundial, pero no el conflicto
\end{itemize}
\end{frame}

\end{document}
